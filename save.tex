\documentclass[11pt, a4paper, twoside]{report}
\usepackage[a4paper]{geometry}
\usepackage{graphicx} 
\usepackage{fontspec}
\usepackage{xunicode}
\usepackage[Sonny]{fncychap}
\usepackage[frenchb]{babel}
\usepackage{fancyhdr}
\usepackage{nameref}
\usepackage{nopageno}
\usepackage{afterpage}
\usepackage{hyperref}
\usepackage{wrapfig}

%Utilise un point comme marqueur de liste plutôt qu'un tiret
\frenchbsetup{StandardLists=true}

%Commande permettant d'ajouter une page vide
\newcommand\blankpage{%
    \null
    \thispagestyle{empty}%
    \addtocounter{page}{-1}%
    \newpage}
   
\newcommand\doubleblankpage{%
    \null
    \thispagestyle{empty}%
    \addtocounter{page}{-1}%
    \newpage%
    \null
    \thispagestyle{empty}%
    \addtocounter{page}{-1}%
    \newpage%
}

%Commande permettant de récupérer le nom du chapitre actuel
\makeatletter
\newcommand*{\currentname}{\@currentlabelname}
\makeatother

%Permet de customiser les entêtes et pieds de page 
\setlength{\headheight}{\textwidth}

\geometry{top=2.5cm, bottom=2cm, inner=3cm, outer=2cm}

\pagestyle{fancy}

\lhead{\nouppercase{\rightmark}}
\chead{}
\rhead{}
\lfoot{}
\cfoot{\thepage}
\rfoot{}
\renewcommand{\headrulewidth}{0.4pt}
\renewcommand{\footrulewidth}{0pt}

\fancypagestyle{plain}{%
  \fancyhf{}%
  \fancyfoot[C]{\thepage}%
  \renewcommand{\headrulewidth}{0pt}
  \renewcommand{\footrulewidth}{0pt}
}

%Commande permettant d'ajouter une ligne horizontale sur la largeur de la page
\newcommand{\HRule}{\rule{\linewidth}{0.5mm}}

%Définition des marges inner et outer de taille différentes pour prendre en compte le recto-verso

\title{Rapport ADONIF}

\begin{document}
%Ne numérote pas les pages
\pagenumbering{gobble}

\afterpage{\blankpage}

\input{./title.tex}

\input{./remerciements.tex}
\input{./resume.tex}

\tableofcontents

\chapter*{Introduction}
\addcontentsline{toc}{chapter}{Introduction}
Dans le cadre de ma formation à l'Institut Universitaire de Technologie de Lille en seconde année de DUT Informatique, j'ai pu réaliser un stage de fin d'études dans un service informatique universitaire du 30 mars 2015 au 26 juin 2015.\newline

Ce stage a été réalisé dans le service "Direction des Systèmes d'Information" (DSI) de la faculté de Pharmacie de l'Université de Lille 2.

La DSI travaille sur un projet d'inventaire et de référentiel des champignons par convention avec la Société Mycologique du Nord de France.
Ce projet consiste à créer une base de données des champignons existants (le référentiel) et une base contenant les localisations des différents champignons dans la région (l'inventaire), permettant de déterminer les espèces présentes.\newline

Dans le cadre de ce projet, les mycologues ont rencontré un problème  sur le terrain lors de leurs récoltes. En effet, ils souhaiteraient pouvoir connaître avec précision le lieu où ils trouvent un champignon, pour pouvoir le retrouver par la suite et établir des statistiques et analyses de récoltes. De plus, ils souhaitent prendre des photos des champignons dans leur environnement naturel, et de pouvoir les partager avec l'ensemble de la communauté mycologique, ce qui n'est pas forcément simple et immédiat avec un appareil photo classique.

La SMNF a donc décidé de rajouter une application mobile connectée à son outil d'inventaire, afin de pouvoir utiliser la puce GPS des smartphones et tablettes, ainsi que les appareils photos de ces équipements. Grâce à cette application, il sera possible d'augmenter la précision géographique de recensement des récoltes et de partager des informations avec la communauté mycologique.
\newline\newline

Ma mission lors de ce stage aura donc été de réaliser cette application mobile, ainsi que ses interactions avec les autres éléments du projet déjà existants (notamment l'inventaire). Ceux-ci ont mené à la création d'une association correspondant à leur utilisation : ADONIF (Association pour le Développement d'Outils Naturalistes et Informatiques pour la Fonge). 

%Démarre la numérotation des pages
\pagenumbering{arabic}
\chapter{Présentation des initiateurs et du projet}
\section{Présentation des initiateurs}
\subsection{L'université Lille 2}
\subsubsection{Historique de l'université}
L’Université Lille 2 Droit et Santé, ou Université Lille 2, est l'une des trois universités de Lille. Elle est issue des anciennes facultés de droit, médecine et pharmacie de l’université de Lille, héritière de l'université de Douai créée en 1559.

L'université Lille 2 est pluridisciplinaire : droit, science politique, santé, gestion et sport. Elle regroupe six unités de formation et de recherche (UFR) : la faculté des sciences juridiques, politiques et sociales, la faculté de médecine, la faculté des sciences pharmaceutiques et biologiques, la faculté de chirurgie dentaire, la faculté des sciences du sport et de l'éducation physique et la faculté de finance, banque, comptabilité (anc. ESA). Elle comprend aussi des IUT (IUT C) et IUP qui sont décentralisés à Roubaix, et le Service de formation permanente (SFP) à proximité de la cité hospitalière. L'Institut d'études politiques de Lille (Sciences-Po Lille), situé à Lille-Moulins, est un établissement rattaché à cette université.

Au niveau recherche, les équipes de Lille 2 sont associées à des organismes comme le Centre National de la Recherche Scientifique (CNRS) et l'Institut national de la santé et de la recherche médicale (INSERM) et en partenariat, dans le domaine de la santé, avec la recherche clinique. Il existe deux écoles doctorales, l'une dans le domaine des sciences juridiques, politiques et de gestion, l'autre en biologie-santé.

\subsubsection{La faculté des sciences pharmaceutiques et biologiques}

La Faculté des Sciences pharmaceutiques et biologiques de Lille 2 est l'héritière de la Faculté mixte de médecine et de pharmacie créée en 1874. La séparation des facultés de médecine et de pharmacie s'est effectuée en 1970, lors de la création de l'université de Lille 2, avec la constitution d'une UER de pharmacie (Unité d'Enseignement et de Rechercheet de Recherche), qui devient autonome en décembre 1980, lors du regroupement des UER médicales au sein de la Faculté de médecine. Elle se trouve près de la cité hospitalière, rue du Professeur-Laguesse.

La faculté de Pharmacie rassemble près de 2000 étudiants, 130 enseignants-chercheurs, 93 personnels techniques et administratifs ainsi que 1000 usagers de la formation continue. Elle dispose également de 14 équipes de recherches, dont 8 labellisées INSERM et CNRS.

Le diplôme de pharmacien ouvre la possibilité d'exercer une centaine de métiers dans des secteurs aussi divers que l'officine, l'industrie pharmaceutique, l'hôpital, la biologie, l'environnement, l'agroalimentaire, la règlementation du médicament... La Faculté de Pharmacie de Lille prépare et accompagne ses étudiants vers ces différents métiers. 

\subsubsection{La DSI}
La Direction des Systèmes d'Information est un Service Général de l'Université placé sous la responsabilité du Président de l'Université et Directeur Général des Services. Il met en œuvre la politique du système d'information, des technologies de l'information et de la communication définie par le Président et le Conseil d'Administration de l'Université dans les domaines de l'enseignement, de la recherche, de la documentation et de la gestion.

La Direction des Systèmes d'Information se compose de quatre pôles de compétences opérationnels :

\begin{description}
\item[Pôle Infrastructures :] Le Pôle Infrastructures a pour missions la gestion, la sécurité et l'optimisation du réseau ainsi que l'assistance aux utilisateurs. Pour l'assistance aux utilisateurs, un HelpDesk ainsi qu'un service d'assistance ont été mis en place pour permettre une prise en charge et une résolution rapide des pannes ou problèmes rencontrés. 
\item[Pôle Informatique des Applications Institutionnelles :] Le Pôle Informatique des Applications Institutionnelles est responsable du fonctionnement (déploiement, paramétrage et maintenance) des applications utilisées par l'Université Lille 2. 
\item[Pôle Urbanisation et Modernisation du SI :] Le Pôle Urbanisation et Modernisation du SI est chargé d'assurer l'évolution du SI pour répondre toujours mieux aux besoins des utilisateurs qu'ils soient étudiants, chercheurs, enseignants ou employés. Il cherche également à optimiser les moyens, les outils et les méthodes de collection, consolidation, modélisation et restitution ds données, matérielles ou immatérielles, de l'Université en vue de fournir une aide à la décision par une vue d'ensemble de l'activité traitée.
\item[Pôle TICE :] Le Pôle TICE fournit une réflexion permanente et un accompagnement des enseignants pour répondre aux attentes des utilisateurs en matière des usages du numérique dans l'enseignement.
\end{description}

Ces quatres pôles sont sous l'égide du direction de la DSI, M. Laforge.


\subsection{SMNF}
La SMNF est une association à buts non lucratifs, créée en 1967, régie par la loi du 1\up{er} juillet 1901. Elle regroupe plus de 200 membres s'intéressant aux champignons pour les reconnaître dans la nature, se familiariser avec leurs propriétés, leur classification, leur rôle, leur protection et leur éventuelle comestibilité. Chacun trouvera au sein de cette association des mycologues amateurs ou professionnels, qui apporteront conseils et connaissances lors des sorties mycologiques.

Cette association organise des sorties sur le terrain, des conférences, des expositions...


\subsection{Adonif}
Il existe une demande d'outils informatiques dans le domaine de la mycologie pour deux objectifs. D'abord, les mycologues souhaitent mieux saisir, exploiter, partager et diffuser leurs connaissances et observations. D'autre part, les pouvoirs publics en ont besoin pour leur gestion de l'environnement et se mettre en conformité avec divers accords internationaux (ex de la convention d'Aarhus qui vise à améliorer l'accès à l'information environnementale).

Il a été constaté que les développements d'outils informatiques autour de la mycologie, lorsqu’ils ne sont pas issus d’initiatives individuelles, sont essentiellement  pris en charge par des associations mycologiques type loi de 1901, avec soutien des pouvoirs publics, soutien financier notamment.

Afin d'éviter les développements simultanés d'outils visant un même but par les diverses associations mycologiques, il a été décidé de créer une association qui servirait à la gestion et la coordination des efforts des développements informatiques concernant la mycologie. Cette association est l'association ADONIF : Association pour le Développement d'Outils Naturalistes et Informatiques pour la Fonge. C'est elle qui porte le projet concernant l'inventaire et le référentiel des champignons.

Cette association a été crée en collaboration entre la SMNF (société mycologique du nord de france), l'association Ascofrance, la SMF (société mycologique de france) et RAFFUT (réseau associatif francophone pour l’étude des fonges ultramarines et tropicales).

Adonif a vocation à être une plate-forme permettant d’assurer, dans la transparence et la cohérence, le développement de certains outils techniques à vocation commune. Elle n’a pas vocation à se constituer en structure mycologique nationale ni en association mycologique ni, plus généralement, à se substituer à des structures existantes.

M. Pierre-Arthur Moreau, Maître de conférence à la faculté de pharmacie est le président de cette association.

\section{Présentation du référentiel}

%Ajouter un titre intermédiaire
\subsection{Nommage d'un champignon}
%taxinomie = espece + classification
Le référentiel d'Adonif a pour vocation de rassembler toutes les informations nomenclaturales et taxinomiques relatives aux espèces de champignons recensées en France.

La taxinomie consiste à définir des taxons (constitués d'espèces, genres, famillles...) qui incluent notamment le rassemblement de synonymes pour un même taxon. La nomenclature définit la manière de nommer les taxon. Elle est régie par le Code International de Nomenclature pour les Algues, Champignons et Plantes. Elle définit notamment le nom d'espèce comme un binôme consitué d'un nom de genre suivi d'un épithète.
Exemple de l'\newline


\begin{minipage}[]{0.49\linewidth}
\begin{description}
\item[Règne: ] Fungi
\item[Phylum: ] Basidiomycota
\item[Classe: ] /
\item[Ordre: ] Polyporales
\item[Famille: ] Meripilaceae
\item[Genre: ] Abortiporus biennis
\item[Epithète: ] lusitanicus
\item[Rang: ] /
\item[Epithète 2: ] /
\item[Autorités: ] (Bull. : Fr.) Singer

\end{description}
\end{minipage}
\begin{minipage}[]{0.49\linewidth}

\flushright
\noindent\includegraphics[width=0.9\textwidth]{abort-biennis.jpg}

\end{minipage}

\section{Présentation de l'inventaire}
Le but de l'inventaire est d'identifier et de localiser les espèces de champignon françaises, ainsi que leur associer des photographies. Cette partie du projet est donc constituée d'une base de données, permettant de stocker ces informations, et d'un site internet permettant de consulter les récoltes, les modifier ou en créer de nouvelles.

La base du référentiel est constituée de plusieurs tables, certaines servant à stocker les récoltes et leurs informations, d'autres permettent de définir les possibilités de remplissage de certains champs. Par exemple, une récolte appartient à une association mycologique. Pour simplifier la saisie et éviter les redondances dans la base à cause des noms mal orthographiés, la base contient une table référençant toutes les associations ayant déjà crée une récolte. Ainsi, l'utilisateur pourra choisir parmi ces associations lors de la saisie de sa récolte, ou en ajouter une nouvelle si la sienne n'est pas présente dans cette liste.

%Voir pour ajouter le MCD en annexes


\subsection{Exemple d'une récolte type}
Pour nous permettre de bien comprendre le fonctionnement et la logique d'une récolte mycologique, mon tuteur, M. Pierre-Arthur Moreau nous a emmené mes collègues et moi effectuer une récolte dans les jardins botaniques de la faculté de pharmacie. Nous allons donc étudier le déroulement de cette récolte, afin de comprendre les différentes problématiques rencontrées par les mycologues lors de leurs travaux.

La première partie consiste à trouver un spécimen à récolter. Une fois un champignon repéré, le mycologue pourra l'examiner dans son environnement et prendre des photos de celui-ci. Lors d'une récolte, il est important de noter diverses informations quant à l'habitat du champignon : sur quoi pousse t'il (son hote), l'étât de cet hôte (mort, moribond ou vivant)...

\chapter{Missions accomplies}
Lors de ce stage, mon objectif était, comme expliqué précédemment, de développer une application mobile permettant d'enregistrer les données des récoltes de champignon sur le terrain. 
Le public visé par l'application étant composé de personnes n'ayant pas forcément l'habitude des nouvelles technologies, il était préférable de n'être obligé de rentrer qu'un minimum d'informations sur l'application, et de pouvoir compléter la récolte en ligne.
\newline
J'ai donc pu diviser deux étapes lors de mon développement : d'abord le développement de l'application mobile, puis celui de pages web, intégrées au site d'inventaire existant, permettant de gérer nos récoltes faites sur mobile.



\section{Développement de l'application mobile}
\subsection{Fonctionnalités du projet}
Après discussions autour du projet avec mes tuteurs, j'ai pu isoler les fonctionnalités que je devais développer pour cette application.

\subsubsection{Gestion des droits}

Le site d'ADONIF dispose d'un système de rangs pour les utilisateurs, chacun correspondant à des droits sur les diverses fonctionnalités offertes. Ainsi, la création de récoltes nécessite d'être un utilisateur enregistré et d'avoir l'autorisation des 
administrateurs de l'inventaire. Pour implémenter cette restriction, nous avons choisi de lancer un écran de connexion au premier démarrage de l'application, permettant à l'utilisateur de s'identifier. On vérifie ensuite si l'utilisateur existe et s'il a les droits. Si c'est le cas, il peut accéder à l'application.

\subsubsection{Insertion des données de récolte}

Pour pouvoir réaliser une récolte, l'application a été dotée d'un formulaire, simplifié par rapport à la création de récolte en ligne, étant donné qu'il est moins pratique de taper du texte sur mobile. Toujours dans le but de simplifier la saisie d'informations, un certain nombre d'éléments se remplit automatiquement : la date, le nom du récolteur (celui de l'utilisateur connecté)...

La saisie d'informations concernant les récoltes nous a amenés à considérer un problème important : la cohérence avec les informations déjà insérées dans la base. En effet, il peut arriver que les utilisateurs fassent des fautes d'orthographe, ne pas tous mettre la même casse... et donc, complexifier les éventuelles recherches effectuées par la suite et amener à de grandes redondances dans la base. 
Pour éviter ces problèmes, l'application récupère de nombreuses informations depuis le référentiel concernant les caractéristiques des champignons par exemple, et les propose en autocomplétion lorsque l'utilisateur entre ses informations. Cela permet également de rendre moins fastidieuse la saisie de données. Par contre, il est tout de même possible d'entrer des valeurs absentes du référentiel, car ce dernier n'est pas exhaustif à l'heure actuelle.

\subsubsection{Echanges d'informations avec le serveur}

La méthode de récupération des informations du référentiel au moment de la saisie était liée à une contrainte : les mycologues vont souvent réaliser des récoltes dans des lieux où l'on n'a pas de réseau, on ne peut donc pas récupérer les informations de manière dynamique depuis le serveur. Il a donc été décidé que les informations utiles du référentiel seraient chargées automatiquement dans une base de données locale au lancement de l'application en même temps que la connexion, qui elle aussi requiert une connexion internet.

Il doit, bien sur, être possible de mettre à jour les données plus tard si l'on souhaite récupérer de nouvelles informations.

De ce fait, l'application est pourvue de capacités de communication avec le serveur de l'association, peut envoyer les données de récolte et recevoir les informations du référentiel.

\subsubsection{Ajout et modification d'informations localement}

L'application est capable de créer des récoltes, mais aussi de modifier celles réalisées. En effet, la méthode de détermination fait que l'on ne peut pas toujours savoir quelle est l'espèce que l'on a trouvée, et qu'il faut faire des études au microscope par exemple. L'utilisateur peut donc rentrer un certain nombre d'informations, puis éditer une récolte plus tard pour compléter ses informations de récolte.

L'application pourra utiliser les outils apportés par l'utilisation d'un smartphone : l'appareil photo et le GPS. 
Les récoltes sont automatiquement géolocalisées lorsqu'elles sont effectuées. On peut également ajouter autant de photos qu'on le souhaite dans une récolte, et celles-ci sont stockées dans la galerie de l'appareil pour pouvoir les récupérer facilement. En effet, les mycologues apprécient de pouvoir constituer leurs archives personnelles, et donc récupérer leurs photos.

zLes utilisateurs devant pouvoir utiliser l'application hors-ligne, nous avons décidé de stocker localement les récoltes lorsqu'elles sont effectuées, puis l'utilisateur peut les uploader lorsqu'il a accès à une connexion et qu'il a inséré toutes les informations qu'il souhaitait.

L'utilisateur doit donc disposer d'un moyen de lister les récoltes insérées, d'uploader les récoltes une par une (au cas où certaines seraient incomplètes) ou toutes en même temps par facilité.

\subsubsection{Contraintes du projet}
Suite à l'étude des fonctionnalités à réaliser, il a fallu choisir quelles technologies et langages utiliser pour développer cette application. 
Lors du choix de la technologie, plusieurs contraintes ont été isolées :
\begin{itemize}
\item Utiliser des technologies gratuites et libres.
\item Pouvoir développer l'application pour android et iOS (afin que tous les membres de l'association puissent l'utiliser).
\item Pouvoir réaliser toutes les fonctionnalités exigées par le projet.
\end{itemize}

Le fait de devoir développer une application fonctionnant sur android et iOS a été une contrainte importante pour le choix de la technologie.

En effet, Adonif souhaitait avoir une version fonctionnelle pour la fin de mon stage. Ainsi, devoir développer l'application deux fois (une fois en Java pour Android, et une fois en Objective C pour iOS) aurait été très long et potentiellement impossible en 3 mois. Nous avons donc décidé d'utiliser une technologie cross-platform, nous permettant de ne réaliser le développement qu'une fois.

\subsection{Les technologies cross-platform }

Il est possible de distinguer plusieurs catégories différentes parmi les framework permettant un développement cross-platform iOS et Android. Nous allons détailler leurs principes, leurs avantages et leurs inconvénients.

Le principe des framework cross-platform est de proposer le développement d'une application, à partir de laquelle ce framework va générer le code correspondant à chacune des plateformes souhaitées. Ainsi, un projet réalisé dans un langage (dépendant du framework) pourra générer un projet Java pour Android, un projet Objective C pour iOS, un projet C\# pour Windows Phone etc.

\subsubsection{Les frameworks web}

Le premier type de framework cross-platform sont les frameworks basés sur les technologies web. Le framework crée pour chaque plateforme visée une application minimale contenant uniquement une vue affichant une page web. Cette page web est l'application du développeur. 
\begin{wrapfigure}{i}{0.25\textwidth}
   \vspace{-20pt}
  \centering
    \includegraphics[width=0.23\textwidth]{cordova_bot.png}
     \vspace{-20pt}
  \caption{Cordova (framework HTML5)}
     \vspace{-10pt}
\end{wrapfigure}
Ceci permet donc de créer un site en HTML5/Javascript une fois, et le faire tourner sur n'importe quel appareil.

Le framework devra donc apporter des fonctions pour accéder aux capacités natives de l'appareil et faire le lien avec celui-ci. On bénéficie donc des avantages du web et des capacités du mobile en même temps. 

Le principal avantage de ce type de technologie est de réutiliser des compétences éventuellement déjà existantes, permettant aux développeurs web de développer des applications mobiles très rapidement. On peut également utiliser les très nombreuses bibliothèques et framework javascript existants, et communiquer avec des serveurs de manière asynchrone très facilement par le biais d'Ajax.

A l'inverse, le fait d'utiliser une webview ralentit le fonctionnement de l'application et est moins fluide que d'utiliser les éléments de GUI natifs si l'interface est trop élaborée. D'autre part, une simple web view ne peut pas utiliser les capacités de l'appareil, et il faudra donc développer des plugins qui créeront du code dans le langage de chaque platform afin d'accéder à ces fonctionnalités (telles que l'appareil photo).

\subsubsection{La génération de code natif}

Autre type de framework, les framework qui permettent un développement natif. Ces frameworks ont choisi un langage de développement (C\#, C++, Java, Python...) et ont développé une librairie permettant d'utiliser les fonctionnalités natives des appareils.
\begin{wrapfigure}{i}{0.25\textwidth}
   \vspace{-12pt}
  \centering
    \includegraphics[width=0.23\textwidth]{qt-logo.jpg}
       \vspace{-10pt}
  \caption{Qt (Framework C++)}
     \vspace{-20pt}
\end{wrapfigure}
Le fonctionnement varie selon les frameworks, mais une des méthodes utilisées est que chaque fonction du framework correspond à un bloc de code dans chaque langage des plateformes supportées. Par exemple, si j'appelle une fonction du framework pour afficher un bouton, les développeurs du framework ont écrit du code qui affiche le bouton dans chacun des langages natifs proposés par le framework. 

La compilation de l'application consistera donc à lire le code du développeur et générer le code correspondant pour chaque plateforme.

L'avantage de cette approche est de conserver les avantages d'un développement natif en étant cross-platform. En effet, avec le langage natif, on peut avoir de meilleures performances, et utiliser les éléments d'interface graphique natifs. Ainsi, l'apparence de l'application est consistante avec les autres applications disponibles.

Par contre, le développeur sera limité par les capacités du framework : toutes les fonctionnalités natives ne sont pas toujours disponibles, il peut y avoir des bugs... et surtout, ces frameworks sont très souvent payants.

%Mettre les pour et contre
\subsubsection{Un cas particulier : Adobe AIR}

Adobe AIR est une technologie permettant de faire fonctionner des applications en ActionScript sur mobile. Habituellement, pour fonctionner, l'actionscript nécessite une machine virtuelle qui exécute le bytecode généré à partir de l'actionscript.
\begin{wrapfigure}{i}{0.25\textwidth}
   \vspace{-12pt}
  \centering
    \includegraphics[width=0.23\textwidth]{adobe-air-logo.jpg}
       \vspace{-5pt}
  \caption{Adobe Air}
     \vspace{-10pt}
\end{wrapfigure}
La machine virtuelle étant développée en langage natif, elle permet d'avoir des performances comparables à celui-ci tout en étant portable sur diverses plateformes. 



C'est par ce biais que les applications en Actionscript fonctionnent sur Android. Par contre, sur iOS, le matériel n'autorise pas l'exécution de cette machine virtuelle. AIR va donc compiler l'application en code natif en temps réel pour qu'il puisse être exécuté sur cette plateforme.

La solution apportée par ce framework fait que l'on peut faire fonctionner une telle application sur tous les appareils qui supportent Adobe AIR, c'est à dire les ordinateurs sous Windows ou OSX, les smartphones Android, iOS et Windows Phone.

\section{Développement du formulaire en ligne}



\chapter{Bilan technique et humain}

\chapter{Conclusion}

\chapter*{Annexes}
\addcontentsline{toc}{chapter}{Annexes}

\input{./references.tex}


\doubleblankpage
\input{./resume.tex}
\end{document}